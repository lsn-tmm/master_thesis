\documentclass{report}

\usepackage{amsmath,amsfonts,amsthm,color}

\begin{document}

\section{Rayleigh-Schr\"{o}dinger perturbation theory}

Consideriamo un operatore Hamiltoniano della forma
\begin{equation}
\hat{H} = \hat{H}_0 + \hat{V}
\quad,
\end{equation}
dove gli autovalori ed autovettori di $H_0$ sono noti,
\begin{equation}
\hat{H}_0 = \sum_\mu E^{(0)}_{\mu} \; | \Phi_\mu \rangle \langle \Phi_\mu |
\quad,
\end{equation}
l'obiettivo della teoria perturbativa e' quello di porgere approssimazioni per gli autovalori ed autovettori di $H$,
\begin{equation}
\hat{H} | \Psi_\mu \rangle = E_\mu \; | \Psi_\mu \rangle
\quad,
\end{equation}
che siano esatte fino ad un certo ordine in $\lambda$. 
Noi siamo in particolare interessati all'espressione dell'energia di stato fondamentale al secondo ordine perturbativo che,
quando lo stato fondamentale e' non degenere, e' data da 
\begin{equation}
E_0 = E^{(0)}_{0} - \sum_{\mu>0} \frac{ | \langle \Psi_\mu^{(0)} | \hat{V} | \Psi_0^{(0)} \rangle |^2 }{ E^{(0)}_{\mu} - E^{(0)}_{0} }
\quad.
\end{equation}
[questo conto ti e' familiare?]

\section{Moller-Plesset perturbation theory}

L'equazione di Schr\"{o}dinger per sistemi di molti elettroni non e' risolubile analiticamente. 
Il punto di partenza della soluzione dell'equazione di Schrodinger e' il metodo di Hartree-Fock, dove ad essere risolta e'
l'equazione di Schrodinger per un operatore a un corpo determinato autoconsistentemente, detto operatore di Fock, e indicato come
\begin{equation}
\hat{F} = \sum_{p \sigma} \, f_p \, \hat{a}_{p\sigma}^\dagger \hat{a}_{p\sigma}
\end{equation}
[questa cosa ti torna?]

La teoria perturbativa di Moller-Plesset riscrive l'Hamiltoniano di Born-Oppenheimer come $\hat{H}_0 + \hat{V}$, dove 
\begin{equation}
\hat{H}_0 = \left[ \hat{F} - \langle \Psi_{HF} | \hat{H} - \hat{F} | \Psi_{HF} \rangle \right]
\end{equation}
e
\begin{equation}
\hat{V} = \hat{H} - \hat{H}_0
\end{equation}
Nella teoria perturbativa di Moller-Plesset, l'energia di stato fondamentale prende la forma seguente
\begin{equation}
E_{MP2} = E_{HF} - \frac{1}{4} \sum_{abij} \frac{ | \langle ij || ab \rangle |^2 }{f_a + f_b - f_i - f_j}
\end{equation}
dove $ab$ sono spin-orbitali virtuali, ed $ij$ sono spin-orbitali occupati.

[ti consiglio questa lezione di un collaboratore di Sherrill, che e' molto completa ed accessibile
\begin{itemize}
\item http://vergil.chemistry.gatech.edu/notes/marshall-MBPT-2010.pdf
\item https://www.youtube.com/watch?v=IYDpLyIWy1c]
\end{itemize}

Benche' semplice da comprendere e implementare, e computazionalmente leggera, la teoria perturbativa di Moller-Plesset
e' notoriamente inaccurata e non convergente [vedi ad esempio questo paper che ne fa un'analisi profonda ed elegante, 
https://aip.scitation.org/doi/10.1063/1.481764]

\section{N-electron valence perturbation theory (NEVPT2)}

La teoria NEVPT2 si basa sempre sulla teoria perturbativa al secondo ordine, ma non utilizza l'operatore di Fock per costruire $\hat{H}_0$. 
Noi seguiamo l'implementazione di Sokolov e Chan, che divide gli orbitali molecolari in tre gruppi,
\begin{itemize}
\item core (doppiamente occupati e chimicamente inattivi, con indici ijkl)
\item attivi (con indici uvwxyz)
\item esterni (con indici abcd)
\end{itemize}
Rispetto a
L'hamiltoniano viene scritto come 




https://pubs.acs.org/doi/pdf/10.1021/acs.jctc.5b01225


\section{NEVPT2 on a quantum computer}

\noindent
Partiamo dalla consueta rappresentazione di $H$,
\begin{equation}
\begin{split}
H &= E_0 + \sum_{pq} h_{pq} E_{pq} + \sum_{prqs} \frac{(pr|qs)}{2} E_{pqsr}
\quad, \\
E_{pq} &= \sum_\sigma c^\dagger_{p\sigma} c_{q\sigma} 
\quad, \\
E_{pqsr} &= \sum_{\sigma\tau} c^\dagger_{p\sigma} c^\dagger_{q\tau} c_{s\tau} c_{r\sigma} 
\end{split}
\end{equation}
Dividiamo la base in IAO e PAO, e definiamo i proiettori
\begin{equation}
\begin{split}
P_{low} &= B_{low} \left[ B_{low}^T S B_{low} \right]^{-1} B_{low}^T S \quad, \\
P_{high} &= B_{high} \left[ B_{high}^T S B_{high} \right]^{-1} B_{high}^T S \quad, 
\end{split}
\end{equation}
Eseguiamo un conto Hartree-Fock nella originaria, quindi nella base IAO. Definiamo
\begin{equation}
\left[ F_{high} \right]_{pq} = P_{high} F P_{high} \quad.
\end{equation}
Questa e' la proiezione dell'operatore di Fock sul sottospazio dei PAO.
Diagonalizziamo l'operatore di Fock $F_{high}$, ed otterremo una base
\begin{equation}
C = (C_{low} , C_{high} )
\end{equation}
in cui possiamo rappresentare l'Hamiltoniano come
\begin{equation}
H = H_{low} + F_{high} + \Big[ H_{low,high} + H_{high} - F_{high} \Big] \equiv H_0 + V
\end{equation}
Supponiamo di aver trovato lo stato fondamentale $\Psi_0$ di $H_{low}$. Allora potremo definire l'operatore
\begin{equation}
W = \Big[ 1 - | \Psi_0 \rangle \langle \Psi_0 | \Big] V
\end{equation}
e l'energia NEVPT2 prende la forma
\begin{equation}
\Delta E = - \int_0^\infty d\tau \langle \Psi_0 | W^\dagger(\tau) W | \Psi_0 \rangle
\end{equation}
Infatti,
\begin{equation}
\begin{split}
\langle \Psi_0 | W^\dagger(\tau) W | \Psi_0 \rangle 
&=
\langle \Psi_0 | V \Big[ 1 - | \Psi_0 \rangle \langle \Psi_0 | \Big] e^{-\tau (H-E_0)} \Big[ 1 - | \Psi_0 \rangle \langle \Psi_0 | \Big] V | \Psi_0 \rangle \\
&=
\Big[ \langle \Psi_0 | V - v_0 \langle \Psi_0 | \Big] \Big[ e^{-\tau (H-E_0)} V | \Psi_0 \rangle - v_0 | \Psi_0 \rangle \Big] \\
&=
\langle \Psi_0 | V e^{-\tau (H-E_0)} V | \Psi_0 \rangle - v_0^2 \\
&=
\sum_{\mu > 0} | \langle \Psi_0 | V | \Psi_\mu \rangle |^2 e^{-\tau (E_\mu-E_0)}
\end{split}
\end{equation}
con $v_0 = \langle \Psi_0 | V | \Psi_0 \rangle$. Quindi,
\begin{equation}
\begin{split}
\Delta E 
&= - \sum_{\mu > 0} | \langle \Psi_0 | V | \Psi_\mu \rangle |^2 \int_0^\infty d\tau e^{-\tau (E_\mu-E_0)} \\
&= - \sum_{\mu > 0}  \frac{ | \langle \Psi_0 | V | \Psi_\mu \rangle |^2 }{ E_\mu - E_0 } 
\end{split}
\end{equation}
{\color{red}
Questa formula e' importante: per calcolare il contributo dei PAOs all'energia, occorrono non solo lo stato fondamental $\Psi_0$ di $H_{low} + F_{high}$
ma anche alcuni stati eccitati. Una volta trovati gli stati eccitati, dovremo prenderne l'overlap con $V | \Psi_0 \rangle$, quindi dovremo anche calcolare
l'azione della perturbazione $V$ su $\Psi_0$. Cominciamo da quest'ultimo conto, perche' ci suggerira' quali stati eccitati saranno importanti nel calcolo
dell'energia finale.
}

\subsection{Azione della perturbazione} 

Calcoliamo ora, usando maiuscole/minuscole per orbitali high/low,
\begin{equation}
\begin{split}
V | \Psi_0 \rangle 
&= \Big[ H_{low,high} + H_{high} - F_{high} \Big] | \Psi_0 \rangle = H_{low,high} | \Psi_0 \rangle \\
&= \sum_{Pq} h_{Pq} E_{Pq} | \Psi_0 \rangle \\
&+ \sum_{Prqs} \frac{(Pr|qs)}{2} E_{Pqsr} | \Psi_0 \rangle 
+ \sum_{prQs} \frac{(pr|Qs)}{2} E_{pQsr} | \Psi_0 \rangle \\
&+ \sum_{PrQs} \frac{(Pr|Qs)}{2} E_{PQsr} | \Psi_0 \rangle \\
&= \sum_{Pq} h_{Pq} E_{Pq} | \Psi_0 \rangle + \sum_{Prqs} \frac{(Pr|qs)+(qs|Pr)}{2} E_{Pqsr} | \Psi_0 \rangle \\
&+ \sum_{PrQs} \frac{(Pr|Qs)}{2} E_{PQsr} | \Psi_0 \rangle \\
&= \sum_{P\sigma} c^\dagger_{P\sigma} \left[ \sum_{q} h_{Pq} c_{q\sigma} + \sum_{\tau qrs} \frac{(Pr|qs)+(qs|Pr)}{2} c^\dagger_{q\tau} c_{s\tau} c_{r\sigma} \right] | \Psi_0 \rangle \\
&+ \sum_{PQ\sigma\tau} c^\dagger_{P\sigma} c^\dagger_{Q\tau}
\left[ \sum_{rs} \frac{(Pr|Qs)}{2} c_{s\tau} c_{r\sigma} \right] | \Psi_0 \rangle \\
&= \sum_{P\sigma} c^\dagger_{P\sigma} A_{P\sigma} | \Psi_0 \rangle + \sum_{PQ\sigma\tau} c^\dagger_{P\sigma} c^\dagger_{Q\tau} B_{P\sigma,Q\tau} | \Psi_0 \rangle
\end{split}
\end{equation}
Quindi abbiamo stati IP ed IP+1 nell'espansione di $V | \Psi_0 \rangle$. Si ha inoltre $v_0 = 0$.

\begin{equation}
\begin{split}
e^{-\tau (H_0 - E_0 )} V | \Psi_0 \rangle 
&= \sum_{P\sigma} e^{- \tau f_P} c^\dagger_{P\sigma} e^{-\tau (H_{low} - E_0 )} A_{P\sigma} | \Psi_0 \rangle \\
&+ \sum_{PQ\sigma\tau} e^{- \tau (f_P+f_Q)} c^\dagger_{P\sigma} c^\dagger_{Q\tau} e^{-\tau (H_{low} - E_0 )} B_{P\sigma,Q\tau} | \Psi_0 \rangle
\end{split}
\end{equation}

\begin{equation}
\begin{split}
\langle \Psi_0 | V e^{-\tau (H_0 - E_0 )} V | \Psi_0 \rangle 
&= \sum_{P\sigma} e^{- \tau f_P} 
\langle \Psi_0 | A_{P\sigma}^\dagger e^{-\tau (H_{low} - E_0 )} A_{P\sigma} | \Psi_0 \rangle \\
&+ \sum_{PQ\sigma\tau} e^{- \tau (f_P+f_Q)} 
\langle \Psi_0 | B_{P\sigma,Q\tau}^\dagger e^{-\tau (H_{low} - E_0 )} B_{P\sigma,Q\tau} | \Psi_0 \rangle \\
&- \sum_{PQ\sigma\tau} e^{- \tau (f_P+f_Q)} 
\langle \Psi_0 | B_{Q\tau,P\sigma}^\dagger e^{-\tau (H_{low} - E_0 )} B_{P\sigma,Q\tau} | \Psi_0 \rangle \\
\end{split}
\end{equation}

Per semplificare l'ultimo termine, osserviamo che 
\begin{equation}
B_{P\sigma,Q\tau} = \sum_{rs} \frac{(Pr|Qs)}{2} c_{s\tau} c_{r\sigma} = - \sum_{rs} \frac{(Qr|Ps)}{2} c_{s\sigma} c_{r\tau} = - 
B_{Q\tau,P\sigma}
\end{equation}
Questo ci porta a
\begin{equation}
\begin{split}
\langle \Psi_0 | V  e^{-\tau (H_0 - E_0 )} V | \Psi_0 \rangle 
&= \sum_{P\sigma} e^{- \tau f_P} 
\langle \Psi_0 | A_{P\sigma}^\dagger e^{-\tau (H_{low} - E_0 )} A_{P\sigma} | \Psi_0 \rangle \\
&+ 2 \sum_{PQ\sigma\tau} e^{- \tau (f_P+f_Q)} 
\langle \Psi_0 | B_{P\sigma,Q\tau}^\dagger e^{-\tau (H_{low} - E_0 )} B_{P\sigma,Q\tau} | \Psi_0 \rangle \\
\end{split}
\end{equation}

{\color{red}
Ci servono gli stati eccitati di $H_{low}$ nei settori ad $N-1$ ed $N-2$ particelle (una particella o due particelle vengono prese dagli IAO e promossa ai PAO.
Questi stati eccitati vengono detti "stati con una buca" oppure "stati con due buche".}


\section{QSE, 1 hole}

{\color{red}
Cominciamo a calcolare gli stati eccitati ad una buca. Non vogliamo risolvere l'equazione di Schr\"{o}dinger tantissime volte, ma connettere lo stato fondamentale
ad $N$ particelle $\Psi_0$ trovato sul computer quantistico agli stati eccitati di interesse con degli operatori di eccitazione dalla struttura "semplice". In altre parole,
\begin{equation}
| \Psi_{\mu,N-1} \rangle = E_\mu | \Psi_0 \rangle
\end{equation}
Formuliamo un Ansatz per questo operatore di eccitazione, come combinazione lineare di operatori di distruzione,
\begin{equation}
E_\mu = \sum_{p\sigma} C_{p\sigma,\mu} c_{p\sigma} %\left( + \sum_{prq \tau} C_{prq\sigma\tau,\mu} c_{p\sigma} c^\dagger_{r\tau} c_{q\tau} + \dots \right)
\end{equation}
Domanda: come si trovano i coefficienti di espansione $C_{p\sigma,\mu}$? Con una tecnica chiamata "quantum subspace expansion". Gli stati
\begin{equation}
| \chi_{p\sigma} \rangle = c_{p\sigma}  | \Psi_0 \rangle
\end{equation}
formano la base di un sottospazio dello spazio di Hilbert. L'equazione di Schrodinger puo' essere proiettata in questo sottospazio e trasformata in un'equazione
agli autovalori.
}
L'equazione di Schrodinger ha la forma sequente,
\begin{equation}
S^\sigma_{pq} = \langle \chi_{p\sigma} | \chi_{q\sigma} \rangle = \langle \Psi_0 | c^\dagger_{p\sigma} c_{q\sigma} | \Psi_0 \rangle
\quad,\quad
H^\sigma_{pq} = \langle \chi_{p\sigma} | H | \chi_{q\sigma} \rangle = \langle \Psi_0 | c^\dagger_{p\sigma} H c_{q\sigma} | \Psi_0 \rangle
\quad,
\end{equation}
con autostati
\begin{equation}
H^\sigma c_\mu^\sigma = E^\sigma_\mu S^\sigma c_\mu^\sigma
\end{equation}
porta a
\begin{equation}
| \Phi_{\mu\sigma} \rangle = \sum_p c_{p\sigma} C_{p\mu}^\sigma | \Psi_0 \rangle 
\end{equation}
e permette di valutare
\begin{equation}
\begin{split}
\langle \Phi_{\mu\sigma} | A_{P\sigma} | \Psi_0 \rangle 
&= \sum_p C_{p\mu}^\sigma \langle \Psi_0 | c^\dagger_{p\sigma} 
\left[ \sum_{q} h_{Pq} c_{q\sigma} + \sum_{\tau qrs} \frac{(Pr|qs)+(qs|Pr)}{2} c^\dagger_{q\tau} c_{s\tau} c_{r\sigma} \right] | \Psi_0 \rangle \\
&=
\sum_{pq} C_{p\mu}^\sigma \rho_{pq}^\sigma h_{Pq} 
+ 
\sum_{pqrs \tau} C_{p\mu}^\sigma \frac{(Pr|qs)+(qs|Pr)}{2} \rho_{prqs}^{\sigma \tau}
\end{split}
\end{equation}

\section{QSE, 2 holes}

{\color{red}
Finiamo calcolando gli stati eccitati a due buche. L'approccio e' identico al caso precedente, solo ci sono piu' indici.
}


\noindent
Similmente, per due eccitazioni,
\begin{equation}
S^{\sigma\tau}_{qp,sr} = \langle \Psi_0 | c^\dagger_{p\sigma} c^\dagger_{q\tau} c_{s\tau} c_{r\sigma} | \Psi_0 \rangle
\quad,\quad
H^{\sigma\tau}_{qp,sr} = \langle \Psi_0 | c^\dagger_{p\sigma} c^\dagger_{q\tau} H c_{s\tau} c_{r\sigma} | \Psi_0 \rangle
\quad,
\end{equation}
porta a
\begin{equation}
| \Phi_{\mu}^{\sigma\tau} \rangle = \sum_{pq} C^{\sigma\tau}_{qp,\mu} c_{q\tau} c_{p\sigma} | \Psi_0 \rangle 
\end{equation}
e quindi a
\begin{equation}
\begin{split}
\langle \Phi_{\mu}^{\sigma\tau} | B_{P\sigma,Q\tau} | \Psi_0 \rangle 
&=
\sum_{prqs} C^{\sigma\tau}_{qp,\mu} \langle \Psi_0 | c^\dagger_{p\sigma} c^\dagger_{q\tau} c_{s\tau} c_{r\sigma} | \Psi_0 \rangle \frac{(Pr|Qs)}{2}  \\
&= \sum_{prqs} C^{\sigma\tau}_{qp,\mu} \frac{(Pr|Qs)}{2} \rho^{\sigma\tau}_{prqs}
\end{split}
\end{equation}


\end{document}