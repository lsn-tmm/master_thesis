\documentclass{article}

\usepackage{amsmath,amsfonts,amsthm,url,bbm,graphicx}
%\usepackage[colorlinks=true]{hyperref}
%\usepackage{listings}
%\usepackage{pythonhighlight}

%\newcommand{\bohr}{\mathrm{a_B}}
%\newcommand{\hartree}{\mathrm{Ha}}
%\newcommand{\pos}[1]{\mathbf{#1}}
%\newcommand{\CRE}[1]{\hat{a}^\dagger_{#1}}
%\newcommand{\DIS}[1]{\hat{a}^{\phantom{\dagger}}_{#1}}

\begin{document}

Esercizi:
\begin{enumerate}
\item Studio dell'energia di stato fondamentale di OH$^{.}$ e OH$^{-}$ con metodi di chimica quantistica: HF, MP2, CCSD, CCSD(T), CASCI, CASSCF (FCI quando possibile) [agosto]
\item Ripetizione dell'esercizio 1 utilizzando gli IAOs [agosto]
\item Studio dell'energia di stato fondamentale di OH$^{.}$ e OH$^{-}$ con algoritmi quantistici di tipo VQE [agosto]
\item $-----------------------------------$
\item Scrivere le equazioni per il trattamento perturbativo delle eccitazioni elettroniche fuori dallo spazio di valenza
\item Analisi del caso di 2 elettroni in N orbitali
\item Implementazione del caso di 2 elettroni in N orbitali -- obiettivo: codice pulito e performante (Davidson)
\item Implementazione del caso generale di M elettroni in N orbitali 
\item Studio dell'energia di stato fondamentale di OH$^{.}$ e OH$^{-}$ con VQE/IAO (i.e. VQE nello spazio di valenza) e trattamento perturbativo delle eccitazioni elettroniche fuori dallo spazio di valenza 
\end{enumerate}

Spiegazione degli acronimi:
\begin{itemize}
\item idrossile: OH
\item IAO: intrinsic atomic orbitals (una base di pochi orbitali di valenza, estratta da una base molto piu' grande)
\item VQE: variational quantum eigensolver
\item Davidson: diagonalizzazione di un'Hamiltoniano $H$ basata su una subroutine $v \mapsto Hv$
\item HF: Hartree-Fock
\item FCI: full configuration interaction: diagonalizzazione esatta
\item MP2: Moller-Plesset perturbation theory: scrivi $H = H_{HF} + (H - H_{HF})$, e teoria perturbativa al secondo ordine in $H - H_{HF}$
\item CCSD: coupled-cluster with singles and doubles $\exp(T) | \Phi_{HF} \rangle$, $T = \sum_{ai} t^a_i c_a^+ c_i + \sum_{abij} t^{ab}_{ij} c_a^+ c_b^+ c_j c_i$
\item CCSD(T): coupled-cluster with singles and doubles and perturbative triples
\item CASCI: complete active space configuration interaction, ossia diagonalizzazione esatta selezionando un insieme di orbitali 
\item CASSCF: complete active space self consistent field, ossia diagonalizzazione esatta ottimizzando un insieme di orbitali 
\end{itemize}


\end{document}
